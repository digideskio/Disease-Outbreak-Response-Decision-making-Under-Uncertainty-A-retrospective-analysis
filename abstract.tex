Despite public health efforts to prevent outbreaks of vaccine-preventable disease, reduction in natural immunity, declines in vaccination rates due to low perceived risk, and regional migration  often results in resurgent outbreaks, as has been observed in various locations around the globe. In response to such events, a common intervention is outbreak response immunization, applied to population subgroups thought to include large numbers of susceptible individuals, in the hope of attenuating the outbreak. However, such interventions are typically applied in the face of considerable uncertainty, with respect to the age distribution of the susceptible population and vital rates for the underlying disease dynamics. This can result in sub-optimal control decisions that ultimately leads to a more severe outbreak event. We propose an integrated modeling approach as a decision support tool, with the aim of providing robust predictions using all information available at key decision points during the outbreak. We apply this framework to data from the 1996-7 measles outbreak in Sao Paulo, Brazil, which occurred despite the application of routine and supplemental vaccination in the years preceding the event. This application integrates prior immunization coverage rates, an age-specific model correcting for differential serological confirmation rates for the virus, a SIR compartment model of disease dynamics, and a data augmentation model for estimating excess susceptible individuals relative to the expected dynamics. We estimated the effective reproductive number based on information that would have been available at one and two months before the actual immunization intervention, and used this estimate to compare the effectiveness of four competing decision alternatives targeting different age groups in the Sao Paulo population. We estimated that a vaccination campaign that solely targets children would not be able to control the outbreak, which was characterized by an unexpected pool of adult infections. Such a result would only have been apparent after combining \textit{a priori} information with monitoring data from early in the outbreak. We recommend a structured, model-based decision-making approach that allows for the efficient incorporation of all relevant information to derive effective strategies the control of infectious disease outbreaks.