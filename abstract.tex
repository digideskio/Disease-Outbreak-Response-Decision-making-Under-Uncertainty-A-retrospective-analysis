Despite public health efforts to prevent outbreaks of vaccine-preventable disease, reduction in natural immunity, declines in vaccination rates due to low perceived risk, and regional migration may result in resurgent outbreaks, as has been observed in various locations around the globe. In response to such events, a common intervention is outbreak response immunization, applied to population subgroups thought to include large numbers of susceptible individuals, in an effort to limit the outbreak. However, such interventions are typically applied in the face of considerable uncertainty with respect to local transmissison rates and the age distribution of the susceptible population that can result in inadequate control measures. We propose an integrated modeling approach as a decision support tool, with the aim of synthesizing historical demographic and vaccination data with near real-time outbreak surveillance to inform decisions about alternative outbreak response targets. We apply this framework to data from the 1996-7 measles outbreak in S\~{a}o Paulo, Brazil, which occurred despite the application of both routine and supplemental vaccination in the preceding years. We develop a model that integrates historical vaccination and demographic data to characterize pre-outbreak risk and real-time outbreak surveillance to fit an age-specific model of differential serological measles confirmation rates from clincally-confirmed cases and a SIR model of disease dynamics.  In addition, we used a data augmentation model to estimate the age distribution of excess susceptibles that could not be accounted for by historical processes. To simulate the information state available to decision-makers, we truncated the outbreak surveillance data to what would have been available at both 1 and 2 months prior to the realized immunization interventions. This allowed us to estimate the effective reproductive number, \(R_e\) and estimate the likelihood that four proposed vaccination age targets for outbreak response could have reduced \(R_e\) to below 1, had they been applied.  We estimated that a vaccination campaign that solely targets children would not be able to control the outbreak, which was characterized by an unexpected pool of adult susceptibles. Such a result would only have been apparent after combining \textit{a priori} information with monitoring data from early in the outbreak. We recommend a structured, model-based decision-making approach that allows for the efficient incorporation of all relevant information to derive effective strategies for the control of infectious disease outbreaks.