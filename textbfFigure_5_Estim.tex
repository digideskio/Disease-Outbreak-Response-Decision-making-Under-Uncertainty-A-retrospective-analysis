\textbf{Figure 5.} Estimated vaccination threshold and the estimated
reduction of susceptibles for alternative age-targeted vaccination
campaigns. Black bars and grey shading give the posterior mean estimate
and 95\% credible interval of the vaccination coverage necessary to
reduce the the estimated RE to 1 for the confirmation bias and clinical
symptoms models fitted to June and July data. The colored bars indicate
the 95\% credible intervals for the estimated reduction in total
suscpetibles due to non-selective vaccination campaigns targeting
individuals under 5 years (red), under 15 years (blue), under 30 years
(dark grey), or children under 5 years plus adults between 20-30 years
(green). If above the range of the estimated vaccination threshold, a
campaign strategy is estimated to be sufficient to reduce RE to below 1.

Regardless of model (age corrected or clinical cases) or observation
point (June or July), the majority of susceptibles greater than 15 years
of age are estimated to be ``excess susceptibles''. The distribution of
excess susceptibles was only constrained to have a Gaussian
distribution; the mean and variance of that distribution was
unconstrained. The best fit model for the July data using the age
corrected model estimates that excess susceptibles were concentrated
between 10-40 years and accounted for 40\% of all susceptibles (Figure 3
; full results for other model/date combinations were similar and are
presented in the supplement). We estimate the impact of these
susceptibles on the 1997 outbreak by calculating the expected
\emph{R\textsubscript{E}} if these susceptibles were removed; for both
models and observation points, the estimate of \emph{R\textsubscript{E}}
in the absence of the excess susceptibles was below 0.5. When the model
was run with excess susceptibles constrained to zero, we were unable to
recover parameter estimates consistent with a measles outbreak. The
estimated mean \emph{R\textsubscript{0}} was over 50, while the
estimated mean \emph{R\textsubscript{E}} was 0.7.