\emph{\textbf{Evaluating Outbreak Response Campaigns}}

The classic result states that a fraction \emph{p\textsubscript{c}
\textgreater{} 1-1/R\textsubscript{e }}of susceptibles must be immunized
(i.e. vaccinated and seroconverted) in order to reduce
\emph{R\textsubscript{e}\textless{}1} (REF: Anderson \& May). Young
children are conventionally targeted in outbreak response vaccination
campaigns because they are, on average, more likely to be susceptible
and they are more likely to experience severe complications due to
measles infection. In populations with a long history of measles
control, the age distribution of susceptibles is frequently wider (REF
Ferrari,Grenfell,Strebel; REF Goodson) and wider campaigns are often
considered to reach larger proportion of susceptibles. As
\emph{R\textsubscript{e}} , and thus the \emph{p\textsubscript{c}},
increases, then one might target a wider age range to increase the
proportion of susceptibles immunized for a given coverage. Here, both
the estimate of \emph{R\textsubscript{e}} and the age distribution of
susceptibles is conditional on both date of the estimate (June or July)
and the model used (all clinical cases or age-corrected). For each
model, we calculate the necessary vaccination threshold assuming a
campaign that achieves 90\% coverage of the target population and 95\%
efficacy. We then evaluate whether there would have been empirical
support on 15 June or 15 July that outbreak response campaigns targeting
individuals from 6m-5y, 6m-15y, 6m-30y, or a mixed strategy targeting
children 6m-5y and adults 20-30y would have met this necessary target.