These differences in $R_E$, with posterior means ranging from 1.02 to 1.25 depending on time (June or July) and observation model (clinical or age-corrected), though small in absolute value, reflect larger practical differences in the necessary vaccination response to stop the outbreak. If we translate these estimates into the minimum vaccination thresholds (assuming campaigns that reach 90\% of the target population and achieves 95\% effectiveness), this corresponds to posterior mean targets of 2 to 20\% of susceptibles that must be vaccinated. If one were to take a conservative estimate and base vaccine thresholds on the 95th percentile of the posterior distribution of $R_E$ estimates, this would correspond to targets of
8 to 24\% (Figure 5). Conditional on a vaccination threshold, we define an ORI campaign as sufficient if it is expected to result in a reduction of
the total susceptible population by at least the threshold amount. Based on the July estimates, both the age confirmation model and the model fit
to all clinical cases estimate that a strategy targeting only children below 5 years of age would not be sufficient to stop the outbreak
(Figure 5). Based on the June estimates, using the age confirmation model, there is evidence that target children under 5 years would not be
sufficient, because the credible interval overlaps the estimated vaccination threshold. The June estimate, fit to the clinical cases only, predicts
that all strategies would be sufficient, because that model significantly under-estimates the required vaccination threshold. In all cases, ORI strategies that target individuals up to 15 or 30 years of age, as well as a mixed strategy that targets children under 5 and adults between 20-30 years of age, are estimated to be sufficient to stop the outbreak.