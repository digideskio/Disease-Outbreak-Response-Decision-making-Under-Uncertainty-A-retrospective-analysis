\subsection{\texorpdfstring{\emph{Disease Dynamics
Model}}{Disease Dynamics Model}}\label{disease-dynamics-model}

\emph{Initial Susceptible Population}

We used the population age structure in 1996 and the history of vaccination, through routine and campaigns, prior to 1997 to estimate the starting number of susceptible individuals in each age class at the beginning of the 1997 outbreak.  We model the number of susceptible individuals in each annual age class at the beginning of 1997, \(S_{a,0}\) as a finite mixture of local susceptibles, \(S_{a,L}\), and excess susceptibles, \(S_{a,E}\) , where excess susceptibles are those that are not consistent with local demographic processes.  We model local susceptibles, \(S_{a,L}\), as 

\begin{eqnarray}
S_a^{(local)} &=& N_a \pi(\text{vacc}) \pi(\text{SIA}) \pi(\text{natural}) \\
&=& N_a (1- 0.85 V_a) \prod_{j=1}^2 \left[1- \psi_j I(a,j) \right] \left[\exp \sum_{y=0}^{a-1} \theta_{a-y} \right].
\end{eqnarray}

Where \(N_a\) is the population size in age class \(a\), \(V_a\) is the routine vaccination coverage experienced by age class a when the were eligible for routine vaccination between 9 and 12 months of age, \(j_{is}\) the coverage of the \(j\)th SIA and \(I_{a,j}\) is an indicator function that is 1 if age class a was eligible for the \(j\)th SIA and 0 otherwise, and \(\theta_i\) the force of infection in year i (i.e. the contribution of natural immunity).  The force of infection is assumed to be 1/A in the absence of vaccination, where \(A=\frac{L}{R_0}\) \cite{Anderson_1981} is the mean age of infection in the absence of vaccination and \(R_0\) is the basic reproduction number. The force of infection \(\theta_i\) in the presence of vaccination is then approximated as \(1-p/A\) \cite{Anderson_1981}.  The basic reproduction number \(R_0\) is estimated below.  We present an analysis of sensitivity to assumptions about historical vaccination coverage in the supplement.

We modeled excess susceptibles as using a data augmentation approach. A latent number of susceptibles, constrained to a value uniformly distributed between zero and 1 million, were added to the model. These excess susceptibles were assigned an age distribution, modeled Gaussian with mean \(\mu_{age}\) and standard deviation \(\sigma_{age}\) and added to the resident susceptibles for a total susceptible count in the Sao Paulo population: \(S_a = S_a^{(local)} + S_a^{(excess)}\). As the number and distribution of excess susceptibles were modeled as latent variables, the estimates of their values were driven by the data themselves, as opposed to being imposed upon the model. This provided the means for accounting for infections that may not have been provided by a local population of susceptibles, but potentially by a pool of migrants from areas of low immunity. As a basis for comparison, we fit a version of the full model assuming no excess susceptibles (i.e. the distribution of susceptibles is forced to be consistent with local demographics), in addition to the one described above, in which excess susceptibles are explicitly added.

The realized susceptibility \(p_s\) was modeled as a normally-distributed random variable on the logit scale, to account for variability due to unmeasured factors:

NEED TO FIX THIS NOTATION

\begin{eqnarray}
\text{logit}(p_s) &\sim& N(\mu_s, \sigma_s) \\
\text{logit}(\pi_s) &=& \mu_s \\
\pi_s &=& \pi^{(natural)} \pi^{(SIA)} \pi^{(vacc)}
\end{eqnarray}


We model the number of measles cases in each age class \(a\) and time step \(t\) as 

\[I_a \sim \text{Poisson}\left(S_{a,t-1} \frac{I_{t-1}B_a}{N} \right)\] 

Where \(S_{a,t-1}\) is the number of susceptibles in age class a at time \(t-1\), \(I_{t-1}\) a row vector of the number of infected individuals in each age class at time \(t-1\), and \(B_a\) is the \(a\)th column of the Who Acquires Infection From Whom (WAIFW) matrix.  We model the WAIFW matrix as an assortative matrix \(B\)

\[B = \left[{
\begin{array}{c}
  {\beta} & {\beta \delta} & \ldots & {\beta \delta^{k-2}} & {\beta \delta^{k-1}}  \\
  {\beta \delta} & {\beta} & \ldots & {\beta \delta^{k-3}} & {\beta \delta^{k-2}} \\
{\beta \delta^2} & {\beta \delta} & \ldots & {\beta \delta^{k-4}} & {\beta \delta^{k-3}}  \\
  \vdots & \vdots & \ddots & \vdots & \vdots \\
  {\beta \delta^{k-1}} & {\beta \delta^{k-2}} & \ldots & {\beta \delta} & {\beta}  \\
\end{array}
}\right]\]

that assumes that interaction among declines exponentially with
difference in age. The basic reproduction number, R\textsubscript{0}, in
the model of the initial population size is taken as the dominant
eigenvalue of the matrix B.