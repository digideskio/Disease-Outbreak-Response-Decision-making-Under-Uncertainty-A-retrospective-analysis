\emph{\textbf{Model Fitting}}

Our model was fit using data truncated at two different dates during the
outbreak, to simulate the information state at June 15 and July 15,
1997. These two dates, which include 3698 and 11,982 cases,
respectively, represent two contrasting levels of available monitoring
data on which to potentially base intervention decisions. Hence,
excluding any potential lags in reporting, the model was fit only to the
information that would have been available at those time points (this
includes the estimation of age-specific confirmation bias). Hence, four
different management scenarios were modeled:

\begin{itemize}
\item
  \begin{quote}
  June 15 information state, adjusted for age-specific confirmation bias
  \end{quote}
\item
  \begin{quote}
  June 15 information state, unadjusted for confirmation bias
  \end{quote}
\item
  \begin{quote}
  July 15 information state, adjusted for age-specific confirmation bias
  \end{quote}
\item
  \begin{quote}
  July 15 information state, unadjusted for confirmation bias
  \end{quote}
\end{itemize}

All models were fit using PyMC 2.3 (REF Patil, Huard, Fonnesbeck 2010),
a software package for the Python programming language that fits
Bayesian statistical models using Markov chain Monte Carlo (MCMC, REF
Brooks et al. 2013) sampling. Each model was sampled for 50,000
iterations using a Metropolis-Hastings sampling algorithm, with the
first 40,000 samples discarded conservatively as a warmup period. Hence,
all inference was based on the final 10,000 samples from each model run.