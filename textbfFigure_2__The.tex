If we assume that the entire pool of susceptibles is from the local population (no excess susceptibles), the best
fit model for 15 June data yields an estimate of
\emph{R\textsubscript{0}} = 57 (95\% credible interval: 54-60). This
estimate is not consistent with previously published estimates of
\emph{R\textsubscript{0}} for measles \cite{EDMUNDS_2000,MOSSONG_2000}; thus we interpret this
as a result of poor model fit and do not consider this model parameterization further (See supplement for full details).

We fit the full epidemic model, including resident and excess
susceptibles, using data through either 15 June or 15 July under an
assumption that either all clinical cases were true measles cases or
that true measles cases were a sub-set of the reported clinical cases,
where the confirmation rate was determined by the age-specific
confirmation model. The estimate of the age distribution of total
susceptibles (both local and excess) was similar at both observation
points (Figure 3a and b). The model fit using the age confirmation model
consistently estimated larger numbers of susceptibles in the {[}0,5) and
{[}20,25{]} year age classes and fewer susceptibles in the {[}15,20)
year age classes (Figure 3a and b). The basic reproduction number, R0,
was estimated from the 15 July data was 12.4 (95\% CI 11.5-13.5) for the
age confirmation model and 12.75 (95\% CI 12-13.5) for model fit to
data using only clinical confirmation. The corresponding estimate using only data available on 15
June data was 11.75 (95\% CI 10-14) and 11.5 (95\% CI 10-12.5) for the
clinical only and age confirmation models respectively.