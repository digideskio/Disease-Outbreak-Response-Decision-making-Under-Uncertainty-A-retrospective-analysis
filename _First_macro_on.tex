% First macro on next line not (yet) supported by LaTeXML: 
% \(\beta^{(conf)} \sim N(\mu,\Sigma)\)

% First macro on next line not (yet) supported by LaTeXML: 
% \(logit(p_{a})\  = \ {{\beta_{a}}^{(conf)}}_{}\)

To estimate the true number of cases for each age group, the estimated
probabilities were used to correct the clinic-reported cases, as modeled
by a binomial distribution, and the total number of cases then
calculated as the sum of this corrected value and the lab-confirmed
cases:

df\({x_{a}}^{(clinic)} \sim Bin({n_{a}}^{(clinic)},p_{a})\)

% First macro on next line not (yet) supported by LaTeXML: 
% \(I_{a} = {x_{a}}^{(clinic)} + {x_{a}}^{(lab)}\)

This was used to inform the estimate of the latent age distribution of
susceptibles, modeled as a Dirichlet distribution:

% First macro on next line not (yet) supported by LaTeXML: 
% \(f^{(age)} \sim Dirichlet(1)\)

% First macro on next line not (yet) supported by LaTeXML: 
% \(I \sim Multinomial(f^{(age)})\)

where \(f^{(age)}\)is a vector of age-specific proportions and 1 a
vector of ones, used as a non-informative prior for the Dirichlet.

\subsection{\texorpdfstring{\emph{Disease Dynamics
Model}}{Disease Dynamics Model}}\label{disease-dynamics-model}

Initial Susceptible Population

We used the population age structure in 1996 and the history of
vaccination, through routine and campaigns, prior to 1997 to estimate
the starting number of susceptible individuals in each age class at the
beginning of the 1997 outbreak. We model the number of susceptible
individuals in each annual age class at the beginning of 1997,
S\textsubscript{a,0} as a finite mixture of local susceptibles,
S\textsubscript{a,L}, and excess susceptibles, S\textsubscript{a,E} ,
where excess susceptibles are those that are not consistent with local
demographic processes. We model local susceptibles,
S\textsubscript{a,L}, as

% First macro on next line not (yet) supported by LaTeXML: 
% \({S_{a}}^{(local)} = N_{a}\pi(vacc)\pi(SIA)\pi(natural)\)

% First macro on next line not (yet) supported by LaTeXML: 
% \({S_{a}}^{(local)} = N_{a}\left( 1 - V_{a}*0.85 \right)\left( 1 - {\psi_{j}I\left( a,j \right)}_{} \right)\left\lbrack \exp\left( \theta_{a - y} \right) \right\rbrack\).

Where N\textsubscript{a} is the population size in age class \emph{a},
V\textsubscript{a} is the routine vaccination coverage experienced by
age class a when the were eligible for routine vaccination between 9 and
12 months of age, \(\psi_{j}\)is the coverage of the
j\textsuperscript{th} SIA and \emph{I\textsubscript{a,j}} is an
indicator function that is 1 if age class \emph{a} was eligible for the
j\textsuperscript{th} SIA and 0 otherwise, and \(\theta_{i}\)is the
force of infection in year \emph{i} (i.e. the contribution of natural
immunity). The force of infection is assumed to be 1/A in the absence of
vaccination, where \(A = \frac{L}{R_{0}}\) (REF Anderson and May) is the
mean age of infection in the absence of vaccination and
R\textsubscript{0} is the basic reproduction number. The force of
infection, \(\theta_{i}\), in the presence of vaccination is then
approximated as 1-p/A (REF Anderson and May). The basic reproduction
number, R\textsubscript{0} is estimated below. We present an analysis of
sensitivity to assumptions about historical vaccination coverage in the
supplement.

We modeled excess susceptibles as using a data augmentation approach. A
latent number of susceptibles, constrained to a value uniformly
distributed between zero and 1 million, were added to the model. These
excess susceptibles were assigned an age distribution, modeled Gaussian
with mean \emph{μ\textsubscript{age}} and standard deviation
\emph{σ\textsubscript{age}} and added to the resident susceptibles for a
total susceptible count in the Sao Paulo population:
\emph{S\textsubscript{a} = S\textsubscript{a}\textsuperscript{(local)} +
S\textsubscript{a}\textsuperscript{(excess)}} . As the number and
distribution of excess susceptibles were modeled as latent variables,
the estimates of their values were driven by the data themselves, as
opposed to being imposed upon the model. This provided the means for
accounting for infections that may not have been provided by a local
population of susceptibles, but potentially by a pool of migrants from
areas of low immunity. As a basis for comparison, we fit a version of
the full model assuming no excess susceptibles (\emph{i.e.} the
distribution of susceptibles is forced to be consistent with local
demographics), in addition to the one described above, in which excess
susceptibles are explicitly added.

The realized susceptibility \(p_{\text{s\ }}\)was modeled as a
normally-distributed random variable on the logit scale, to account for
variability due to unmeasured factors:

NEED TO FIX THIS NOTATION

% First macro on next line not (yet) supported by LaTeXML: 
% \(logit(p_{s})\  \sim N(\mu_{s},{\sigma_{s})}^{}\)

% First macro on next line not (yet) supported by LaTeXML: 
% \(logit(\pi_{s})\  = \ \mu_{s}\)

% First macro on next line not (yet) supported by LaTeXML: 
% \(\pi_{s} = \pi^{(natural)}\pi^{(SIA)}\pi^{(vacc)}\)

We model the number of measles cases in each age class, \emph{a}, and
time step, \emph{t}, as

% First macro on next line not (yet) supported by LaTeXML: 
% \(I_{a} \sim\)Poisson\(\left( {S_{a,t - 1}\frac{I_{t - 1}B_{a}}{N}}_{} \right)\)

Where S\textsubscript{a,t-1} is the number of susceptibles in age class
a at time \emph{t-1, I\textsubscript{t-1}} a row vector of the number of
infected individuals in each age class at time t-1, and
B\textsubscript{a} is the a\textsuperscript{th} column of the Who
Acquires Infection From Whom (WAIFW) matrix. We model the WAIFW matrix
as an assortative matrix, B,