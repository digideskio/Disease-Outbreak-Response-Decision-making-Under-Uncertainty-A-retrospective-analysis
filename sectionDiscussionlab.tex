\section{Discussion}\label{discussion}

The 1997 measles outbreak in Sao Paulo was unexpected, having followed several years of high routine vaccination coverage, SIAs, and relatively low
incidence. Further, the age distribution of cases, with a secondary mode among adults, had not been observed in previous outbreaks. Thus, while
historical precedent often serves as a guide for outbreak response, in this case, historical precedent would have greatly under-estimated
outbreak risk and the vaccination targets necessary to limit the outbreak. Here we have presented a novel approach for integrating
real-time outbreak surveillance into the evaluation of an evolving outbreak in order to evaluate candidate response strategies. In doing
so, we have developed a model for interpreting clinical measles surveillance that acknowledges that the correlation between clinical
measles symptoms and lab confirmation of positive measles IgM serology is age-specific. Further, we have shown that relying on clinical
confirmation alone can significantly bias inference about transmission rate ($R_E$) and the minimal vaccination targets to stop an outbreak.

In the case of Sao Paulo in 1997, estimates of $R_E$ and the likelihood that different age-targeted vaccination campaigns would meet the necessary immunization threshold were similar (\emph{e.g.} under-5y campaigns (and perhaps under-15y) were estimated to be insufficient to limit the outbreak), regardless of whether confirmation bias was corrected, when fit to data on 15 July. Thus, while our proposed model accounting for age-specific bias in serologic confirmation explicitly estimates the uncertainty in clinical diagnosis, it results in little practical difference in the interpretation of risk \(R_E\) or candidate interventions on 15 July. However, using only the data available on 15 June, estimates based only on clinical confirmation data would have grossly under-estimated risk and over-estimated the benefit of a vaccination campaign targeting children below 5 years of age. 

Although outbreak risk can be evaluated \emph{a priori}, outbreaks themselves are often the first indication of the build-up of susceptibles or gaps in immunity. In 1997, the age distribution of cases in Sao Paulo indicated a dangerous gap in immunity among individuals between 15-35 years of
age. The SIA conducted in 1987, targeting children below 14 years of age, would be expected to have immunized individuals below 23 years of
age, and those older than 23 years would have been born prior to a national immunization system in Brazil and would be expected
to have experienced natural infection during their childhood. We estimated that excess susceptibles between 15-35 years of age may have accounted for 66\% of all susceptibles during the 1997 outbreak. We term these as \emph{excess} susceptibles because they are
excess relative to the expected age distribution of susceptibles based on historical rates of natural infection, routine vaccination, and SIAs. We are unable to positively identify the source of these excess susceptibles; they may have been the result of over-estimating the coverage of previous
vaccination programs, or migrants from low coverage or low transmission risk areas that were unlikely to be exposed to vaccination or natural infection. While the former explanation is possible, insufficient vaccination coverage would be expected to result in more circulating infection, which would still likely result exposure to natural infection, and thus immunity, by adulthood. The latter explanation requires that individuals were recent immigrants to Sao Paulo and had not been exposed to either vaccination or natural infection as children in the region that the emigrated from. Measles rarely persists endemically in small populations below some critical community size \cite{Conlan_2007, Keeling_1997}; thus it is possible that recent immigrants from small villages might have not been exposed to natural infection. \citet{Camargo_2000} conducted a case-control trial after the 1997 outbreak and found that recent immigration to Sao Paulo was a significant risk factor for measles infection during the outbreak. Further, immigration rates into Sao Paulo in 1991 were highest among individuals between 15 and 30 years of age  \cite{de_Moraes_2016}, which is consistent with the age distribution of the excess susceptibles estimated by our models. While this does not confirm that
immigration or gaps in prior immunization were the source of the adult susceptibles during the 1997 outbreak, this analysis does suggest that
these adult susceptibles may have played a significant role in the outbreak; absent the excess susceptibles, $R_E$ at the start of the outbreak would have been comfortably less than 1. Other recent measles outbreaks have exhibited this same age-profile, with an unexpectedly large number of adult cases (\emph{e.g.} Malawi \cite{Minetti_2013}, Mongolia (ref: website above), China \cite{Zheng_2015}). Thus, strategies for monitoring and targeting immunity gaps in adults may be useful in preventing future outbreaks. Moreover, outbreak response strategies should consider adult-targeted vaccination when surveillance indicates a large number of adult susceptibles.

Though our models account for the age distribution of susceptibles, we make very simplistic assumptions about age-specific transmission; namely that
within age-class transmission is the same for all ages, and between age-class transmission decays exponentially with difference in ages. Considerable recent work has shown that age-specific mixing rates are likely to vary considerably and may be culturally specific \cite{Mossong_2008}. It is possible that higher contact (and thus, transmission) rates among adults means that adult susceptibles disproportionately contributed to this outbreak. Fitting an age-specific transmission matrix with differential age-specific rates would have greatly inflated the model complexity and is beyond the scope of this work \citep{327793}.

There are several potential explanations for the age-specific serological confirmation bias for cases with clinical symptoms. There are many
etiologies that may generate fever and rash symptoms in young children  \cite{Hutchins_2004,Ho_2014,GUY_2004,31c964}, which would increase the rate of false
positives in children based on clinical symptoms alone. Further, young children may be brought to clinic shortly after the onset of symptoms,
when IgM titers may not yet have reached detectable levels. Regardless of the cause of this bias, the result is that assessing the
overall age distribution of cases may tend to underestimate the role of adults if based on clinical confirmation alone. Further, as the age
distribution of cases presenting with symptoms may change over the course of an outbreak (give examples from this outbreak), clinical cases
may reflect time-varying confirmation and bias estimates of transmission rates. If true, by correcting for age-specific confirmation bias, we also
necessarily correct for temporal variation in false positives. We see the effect of this in the higher estimated \(R_E\) using the confirmation
bias corrected model with the 15 June data, relative to that of the clinical confirmation model; though the model predicts fewer true measles cases (\emph{i.e}. false positives are removed), the exponential increase from the start of the outbreak to 15 June is steeper and this estimate is consistent with the higher \(R_E\) estimate from the longer time series available on 15 July.

This work highlights the value of an integrated approach to model-based inference and prediction in supporting decision-making for the control of measles outbreaks. Faciliated by a Bayesian hierarchical model structure, we were able to account for all important sources of uncertainty, including the structure of the disease dynamics model, partial observability, and uncertainties regarding vaccine efficacy and vaccination coverage, to produce realistic estimates of susceptibility, outbreak intensity and intervention effectiveness conditional on information that would have been available at the time. The model is comprised of component sub-models, which are linked via conditional probability statements; this results in final estimates of parameters that fully incorporate the uncertainty in their dependent variables, resulting in predictions that more fully account for the lack of complete information in decision-making. Despite the apparent complexity of the integrated model, it was relatively straightforward to implement and estimate using readily-avaialble, open-source software, which allowed us to easily evaluate some of our key assumptions regarding model structure and prior information in terms of their influence on model estimates and predictions. 

\section{Discussion}\label{discussion}

The 1997 measles outbreak in Sao Paulo was unexpected, having followed several years of high routine vaccination coverage, SIAs, and relatively low
incidence. Further, the age distribution of cases, with a secondary mode among adults, had not been observed in previous outbreaks. Thus, while
historical precedent often serves as a guide for outbreak response, in this case, historical precedent would have greatly under-estimated
outbreak risk and the vaccination targets necessary to limit the outbreak. Here we have presented a novel approach for integrating
real-time outbreak surveillance into the evaluation of an evolving outbreak in order to evaluate candidate response strategies. In doing
so, we have developed a model for interpreting clinical measles surveillance that acknowledges that the correlation between clinical
measles symptoms and lab confirmation of positive measles IgM serology is age-specific. Further, we have shown that relying on clinical
confirmation alone can significantly bias inference about transmission rate ($R_E$) and the minimal vaccination targets to stop an outbreak.

In the case of Sao Paulo in 1997, estimates of $R_E$ and the likelihood that different age-targeted vaccination campaigns would meet the necessary immunization threshold were similar (\emph{e.g.} under-5y campaigns (and perhaps under-15y) were estimated to be insufficient to limit the outbreak), regardless of whether confirmation bias was corrected, when fit to data on 15 July. Thus, while our proposed model accounting for age-specific bias in serologic confirmation explicitly estimates the uncertainty in clinical diagnosis, it results in little practical difference in the interpretation of risk \(R_E\) or candidate interventions on 15 July. However, using only the data available on 15 June, estimates based only on clinical confirmation data would have grossly under-estimated risk and over-estimated the benefit of a vaccination campaign targeting children below 5 years of age. 

Although outbreak risk can be evaluated \emph{a priori}, outbreaks themselves are often the first indication of the build-up of susceptibles or gaps in immunity. In 1997, the age distribution of cases in Sao Paulo indicated a dangerous gap in immunity among individuals between 15-35 years of
age. The SIA conducted in 1987, targeting children below 14 years of age, would be expected to have immunized individuals below 23 years of
age, and those older than 23 years would have been born prior to a national immunization system in Brazil and would be expected
to have experienced natural infection during their childhood. We estimated that excess susceptibles between 15-35 years of age may have accounted for 66\% of all susceptibles during the 1997 outbreak. We term these as \emph{excess} susceptibles because they are
excess relative to the expected age distribution of susceptibles based on historical rates of natural infection, routine vaccination, and SIAs. We are unable to positively identify the source of these excess susceptibles; they may have been the result of over-estimating the coverage of previous
vaccination programs, or migrants from low coverage or low transmission risk areas that were unlikely to be exposed to vaccination or natural infection. While the former explanation is possible, insufficient vaccination coverage would be expected to result in more circulating infection, which would still likely result exposure to natural infection, and thus immunity, by adulthood. The latter explanation requires that individuals were recent immigrants to Sao Paulo and had not been exposed to either vaccination or natural infection as children in the region that the emigrated from. Measles rarely persists endemically in small populations below some critical community size \cite{Conlan_2007, Keeling_1997}; thus it is possible that recent immigrants from small villages might have not been exposed to natural infection. \citet{Camargo_2000} conducted a case-control trial after the 1997 outbreak and found that recent immigration to Sao Paulo was a significant risk factor for measles infection during the outbreak. Further, immigration rates into Sao Paulo in 1991 were highest among individuals between 15 and 30 years of age  \cite{de_Moraes_2016}, which is consistent with the age distribution of the excess susceptibles estimated by our models. While this does not confirm that
immigration or gaps in prior immunization were the source of the adult susceptibles during the 1997 outbreak, this analysis does suggest that
these adult susceptibles may have played a significant role in the outbreak; absent the excess susceptibles, $R_E$ at the start of the outbreak would have been comfortably less than 1. Other recent measles outbreaks have exhibited this same age-profile, with an unexpectedly large number of adult cases (\emph{e.g.} Malawi \cite{Minetti_2013}, Mongolia (ref: website above), China \cite{Zheng_2015}). Thus, strategies for monitoring and targeting immunity gaps in adults may be useful in preventing future outbreaks. Moreover, outbreak response strategies should consider adult-targeted vaccination when surveillance indicates a large number of adult susceptibles.

Though our models account for the age distribution of susceptibles, we make very simplistic assumptions about age-specific transmission; namely that
within age-class transmission is the same for all ages, and between age-class transmission decays exponentially with difference in ages. Considerable recent work has shown that age-specific mixing rates are likely to vary considerably and may be culturally specific \cite{Mossong_2008}. It is possible that higher contact (and thus, transmission) rates among adults means that adult susceptibles disproportionately contributed to this outbreak. Fitting an age-specific transmission matrix with differential age-specific rates would have greatly inflated the model complexity and is beyond the scope of this work. \cite{327793}

There are several potential explanations for the age-specific serological confirmation bias for cases with clinical symptoms. There are many
etiologies that may generate fever and rash symptoms in young children  \cite{Hutchins_2004,Ho_2014,GUY_2004,31c964}, which would increase the rate of false
positives in children based on clinical symptoms alone. Further, young children may be brought to clinic shortly after the onset of symptoms,
when IgM titers may not yet have reached detectable levels. Regardless of the cause of this bias, the result is that assessing the
overall age distribution of cases may tend to underestimate the role of adults if based on clinical confirmation alone. Further, as the age
distribution of cases presenting with symptoms may change over the course of an outbreak (give examples from this outbreak), clinical cases
may reflect time-varying confirmation and bias estimates of transmission rates. If true, by correcting for age-specific confirmation bias, we also
necessarily correct for temporal variation in false positives. We see the effect of this in the higher estimated \(R_E\) using the confirmation
bias corrected model with the 15 June data, relative to that of the clinical confirmation model; though the model predicts fewer true measles cases (\emph{i.e}. false positives are removed), the exponential increase from the start of the outbreak to 15 June is steeper and this estimate is consistent with the higher \(R_E\) estimate from the longer time series available on 15 July.

This work highlights the value of an integrated approach to model-based inference and prediction in supporting decision-making for the control of measles outbreaks. Faciliated by a Bayesian hierarchical model structure, we were able to account for all important sources of uncertainty, including the structure of the disease dynamics model, partial observability, and uncertainties regarding vaccine efficacy and vaccination coverage. This, in turn, produced estimates of susceptibility, outbreak intensity and intervention effectiveness conditional only on information that would have been available at the time. The model is comprised of component sub-models, which are linked explicitly via conditional probability statements; this results in parameter estimates that fully incorporate the uncertainty in their dependent variables, ultimately yielding predictions that more fully account for the lack of complete information at hand for decision-making. For example, the likely presence of a pool of excess susceptibles in the population was not evident based on historical immunization records, but was made apparent only when this information was integrated with outbreak monitoring data, via realistic model of measles dynamics. From this, the model predicted that only campaigns targeting adults as well as children were likely to be effective in stopping the outbreak, and this was evident from data available as early as July 15. Despite the apparent complexity of the integrated model, it was straightforward to implement and fit using readily-available, open-source software, which allowed us to easily evaluate key assumptions regarding model structure and prior information in terms of their influence on model estimates and predictions. We believe this approach holds promise as a decision support tool for a variety of disease outbreak scenarios, by efficiently incorporating all available information and incorporating monitoring information to provide robust decision support under uncertainty.


\section{Discussion}\label{discussion}

The 1997 measles outbreak in Sao Paulo was unexpected, having followed several years of high routine vaccination coverage, SIAs, and relatively low
incidence. Further, the age distribution of cases, with a secondary mode among adults, had not been observed in previous outbreaks. Thus, while
historical precedent often serves as a guide for outbreak response, in this case, historical precedent would have greatly under-estimated
outbreak risk and the vaccination targets necessary to limit the outbreak. Here we have presented a novel approach for integrating
real-time outbreak surveillance into the evaluation of an evolving outbreak in order to evaluate candidate response strategies. In doing
so, we have developed a model for interpreting clinical measles surveillance that acknowledges that the correlation between clinical
measles symptoms and lab confirmation of positive measles IgM serology is age-specific. Further, we have shown that relying on clinical
confirmation alone can significantly bias inference about transmission rate ($R_E$) and the minimal vaccination targets to stop an outbreak.

In the case of Sao Paulo in 1997, estimates of $R_E$ and the likelihood that different age-targeted vaccination campaigns would meet the necessary immunization threshold were similar (\emph{e.g.} under-5y campaigns (and perhaps under-15y) were estimated to be insufficient to limit the outbreak), regardless of whether confirmation bias was corrected, when fit to data on 15 July. Thus, while our proposed model accounting for age-specific bias in serologic confirmation explicitly estimates the uncertainty in clinical diagnosis, it results in little practical difference in the interpretation of risk \(R_E\) or candidate interventions on 15 July. However, using only the data available on 15 June, estimates based only on clinical confirmation data would have grossly under-estimated risk and over-estimated the benefit of a vaccination campaign targeting children below 5 years of age. 

Although outbreak risk can be evaluated \emph{a priori}, outbreaks themselves are often the first indication of the build-up of susceptibles or gaps in immunity. In 1997, the age distribution of cases in Sao Paulo indicated a dangerous gap in immunity among individuals between 15-35 years of
age. The SIA conducted in 1987, targeting children below 14 years of age, would be expected to have immunized individuals below 23 years of
age, and those older than 23 years would have been born prior to a national immunization system in Brazil and would be expected
to have experienced natural infection during their childhood. We estimated that excess susceptibles between 15-35 years of age may have accounted for 66\% of all susceptibles during the 1997 outbreak. We term these as \emph{excess} susceptibles because they are
excess relative to the expected age distribution of susceptibles based on historical rates of natural infection, routine vaccination, and SIAs. We are unable to positively identify the source of these excess susceptibles; they may have been the result of over-estimating the coverage of previous
vaccination programs, or migrants from low coverage or low transmission risk areas that were unlikely to be exposed to vaccination or natural infection. While the former explanation is possible, insufficient vaccination coverage would be expected to result in more circulating infection, which would still likely result exposure to natural infection, and thus immunity, by adulthood. The latter explanation requires that individuals were recent immigrants to Sao Paulo and had not been exposed to either vaccination or natural infection as children in the region that the emigrated from. Measles rarely persists endemically in small populations below some critical community size \cite{Conlan_2007, Keeling_1997}; thus it is possible that recent immigrants from small villages might have not been exposed to natural infection. \citet{Camargo_2000} conducted a case-control trial after the 1997 outbreak and found that recent immigration to Sao Paulo was a significant risk factor for measles infection during the outbreak. Further, immigration rates into Sao Paulo in 1991 were highest among individuals between 15 and 30 years of age  \cite{de_Moraes_2016}, which is consistent with the age distribution of the excess susceptibles estimated by our models. While this does not confirm that
immigration or gaps in prior immunization were the source of the adult susceptibles during the 1997 outbreak, this analysis does suggest that
these adult susceptibles may have played a significant role in the outbreak; absent the excess susceptibles, $R_E$ at the start of the outbreak would have been comfortably less than 1. Other recent measles outbreaks have exhibited this same age-profile, with an unexpectedly large number of adult cases (\emph{e.g.} Malawi \cite{Minetti_2013}, Mongolia (ref: website above), China \cite{Zheng_2015}). Thus, strategies for monitoring and targeting immunity gaps in adults may be useful in preventing future outbreaks. Moreover, outbreak response strategies should consider adult-targeted vaccination when surveillance indicates a large number of adult susceptibles.

Though our models account for the age distribution of susceptibles, we make very simplistic assumptions about age-specific transmission; namely that
within age-class transmission is the same for all ages, and between age-class transmission decays exponentially with difference in ages. Considerable recent work has shown that age-specific mixing rates are likely to vary considerably and may be culturally specific \cite{Mossong_2008}. It is possible that higher contact (and thus, transmission) rates among adults means that adult susceptibles disproportionately contributed to this outbreak. Fitting an age-specific transmission matrix with differential age-specific rates would have greatly inflated the model complexity and is beyond the scope of this work. \cite{327793}

There are several potential explanations for the age-specific serological confirmation bias for cases with clinical symptoms. There are many
etiologies that may generate fever and rash symptoms in young children  \cite{Hutchins_2004,Ho_2014,GUY_2004,31c964}, which would increase the rate of false
positives in children based on clinical symptoms alone. Further, young children may be brought to clinic shortly after the onset of symptoms,
when IgM titers may not yet have reached detectable levels. Regardless of the cause of this bias, the result is that assessing the
overall age distribution of cases may tend to underestimate the role of adults if based on clinical confirmation alone. Further, as the age
distribution of cases presenting with symptoms may change over the course of an outbreak (give examples from this outbreak), clinical cases
may reflect time-varying confirmation and bias estimates of transmission rates. If true, by correcting for age-specific confirmation bias, we also
necessarily correct for temporal variation in false positives. We see the effect of this in the higher estimated \(R_E\) using the confirmation
bias corrected model with the 15 June data, relative to that of the clinical confirmation model; though the model predicts fewer true measles cases (\emph{i.e}. false positives are removed), the exponential increase from the start of the outbreak to 15 June is steeper and this estimate is consistent with the higher \(R_E\) estimate from the longer time series available on 15 July.

This work highlights the value of an integrated approach to model-based inference and prediction in supporting decision-making for the control of measles outbreaks. Faciliated by a Bayesian hierarchical model structure, we were able to account for all important sources of uncertainty, including the structure of the disease dynamics model, partial observability, and uncertainties regarding vaccine efficacy and vaccination coverage. This, in turn, produced estimates of susceptibility, outbreak intensity and intervention effectiveness conditional only on information that would have been available at the time. The model is comprised of component sub-models, which are linked explicitly via conditional probability statements; this results in parameter estimates that fully incorporate the uncertainty in their dependent variables, ultimately yielding predictions that more fully account for the lack of complete information at hand for decision-making. For example, the likely presence of a pool of excess susceptibles in the population was not evident based on historical immunization records, but was made apparent only when this information was integrated with outbreak monitoring data, via realistic model of measles dynamics. From this, the model predicted that only campaigns targeting adults as well as children were likely to be effective in stopping the outbreak, and this was evident from data available as early as July 15. Despite the apparent complexity of the integrated model, it was straightforward to implement and fit using readily-available, open-source software, which allowed us to easily evaluate key assumptions regarding model structure and prior information in terms of their influence on model estimates and predictions. We believe this approach holds promise as a decision support tool for a variety of disease outbreak scenarios, by efficiently incorporating all available information and incorporating monitoring information to provide robust decision support under uncertainty.


